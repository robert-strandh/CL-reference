\chapter{Data types}
\label{chap-data-types}

\section{\commonlisp{} is dynamically typed}

\commonlisp{} is a \emph{dynamically-typed language}, by which we mean
that every \emph{object} that is manipulated by some program in the
language carries with it associated information about its \emph{type},
and that in the general case, a \emph{variable} in \commonlisp{} can
hold an object of any type, and even objects of different type at
different times during program execution.  Dynamic typing is the
opposite of \emph{static typing} in which the type of an object is
known only because the \emph{type of variables that have that object
  as a value} is known at \emph{compile time}.  In the case of static
typing, the compiler must acquire this type information either by
explicit type declarations or by a process known as \emph{type
  inference}.

The contrast between static typing and dynamic typing should not be
confused with the contrast between \emph{strong typing} and \emph{weak
  typing}.  A language is \emph{strongly typed} if it is impossible to
write a program that mistakenly interprets an object of one type as
being an object of a different type.  Conversely, a \emph{weakly
  typed} language is one where such type confusion is possible.

While the \commonlisp{} standard allows for implementation to be
weakly typed, most implementations are strongly typed, at least when
the \texttt{safety} level is high.  Weak typing occurs because, in
many cases, the standard allows for ``the consequences to be
undefined'' when there is a violation of the type of an object that is
passed as an argument to a standard operator.  A program that allows
for such a situation is, however, non conforming.
