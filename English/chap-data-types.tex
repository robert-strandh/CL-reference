\chapter{Data types}
\label{chap-data-types}

\section{\commonlisp{} is dynamically typed}

\commonlisp{} is a \emph{dynamically-typed language}, by which we mean
that every \emph{object} that is manipulated by some program in the
language carries with it associated information about its \emph{type},
and that in the general case, a \emph{variable} in \commonlisp{} can
hold an object of any type, and even objects of different type at
different times during program execution.  Dynamic typing is the
opposite of \emph{static typing} in which the type of an object is
known only because the \emph{type of variables that have that object
  as a value} is known at \emph{compile time}.  In the case of static
typing, the compiler must acquire this type information either by
explicit type declarations or by a process known as \emph{type
  inference}.

The contrast between static typing and dynamic typing should not be
confused with the contrast between \emph{strong typing} and \emph{weak
  typing}.  A language is \emph{strongly typed} if it is impossible to
write a program that mistakenly interprets an object of one type as
being an object of a different type.  Conversely, a \emph{weakly
  typed} language is one where such type confusion is possible.

While the \commonlisp{} standard allows for implementation to be
weakly typed, most implementations are strongly typed, at least when
the \texttt{safety} level is high.  Weak typing occurs because, in
many cases, the standard allows for ``the consequences to be
undefined'' when there is a violation of the type of an object that is
passed as an argument to a standard operator.  A program that allows
for such a situation is, however, non conforming.

\section{Type system}
\label{sec-type-system}

Contrary to statically-typed languages, \commonlisp{} has a type
system that is both trivial in some respects, and very sophisticated
in other respects.

The type system is trivial because a type is defined to be a
\emph{possibly infinite set of objects}.  The languages allows for
various ways to specify such sets, including:

\begin{itemize}
\item enumerating individual objects,
\item giving ranges of real numbers,
\item supplying the name of a \emph{class},
\item indicating types of contained objects for objects such as
  \texttt{cons} cells and arrays,
\item naming a \emph{characteristic function} for the set,
\item using set operations with other sets as arguments.
\end{itemize}

At the same time, the type system is very sophisticated, because it
allows the programmer to specify types that most languages have no way
of expressing, such as:

\begin{itemize}
\item all integers except 0,
\item all primes,
\item any object that is either a number or a person,
\item any object that is not a football.
\end{itemize}

The price to pay for this kind of power is that the \commonlisp{} type
system is not \emph{decidable} in general.

\section{Overview of standard data types}
\label{sec-overview-of-standard-data-types}

\subsection{Atoms and lists}
\label{sec-atoms-and-lists}

It is common to hear people who know very little about \commonlisp{}
denigrate the language because, as everybody knows, the only data
types that the language defines are \emph{atoms} and \emph{lists}. 

These people are perfectly right that these are the only two data
types in the language, except that what they most often do not know is
that \texttt{atom} is the name of the union of all \commonlisp{} types
except the type named \texttt{cons}.  As such, the data type
\texttt{atom} includes objects such as symbols, arrays, streams, class
instances, classes, packages, structures, etc.

The reason for using this fact to reject \commonlisp{} very likely
originates in very early versions of LISP%
\footnote{Notice the spelling of LISP with all capital letters.} in
which the type named \texttt{atom} only contained very simple
objects such as \emph{symbols} and \emph{numbers}.
