\chapter{The concept of a list}
\label{chap-list-concept}

In \cl{}, the word \emph{list} is used very broadly as a structure
made up of \texttt{cons} cells.%
\footnote{The name \texttt{cons} is also the name of the function for
  creating \texttt{cons} cells, and it is an abbreviation of the word
  \emph{construct.}}  A \texttt{cons} cell can be thought of as an
object consisting of two \emph{fields} called the \texttt{car} field
and the \texttt{cdr} field respectively.%
\footnote{The names \texttt{car} and \texttt{cdr} come from the
  machine instructions on one of the first computers on which \lisp{}
  was implemented.  These instructions would return the \emph{contents
    of the address part of the register} and the \emph{contents of the
    decrement part of the register} respectively.}
\refFig{fig-cons-cell} illustrates a \texttt{cons} cell.

\begin{figure}
\begin{center}
\inputfig{fig-cons-cell.pdf_t}
\end{center}
\caption{\label{fig-cons-cell}
Representation of a \texttt{cons} cell.}
\end{figure}

