\chapter{The concept of a list}
\label{chap-list-concept}

In \cl{}, the word \emph{list} is used very broadly as a structure
made up of \texttt{cons} cells.%
\footnote{The name \texttt{cons} is also the name of the function for
  creating \texttt{cons} cells, and it is an abbreviation of the word
  \emph{construct.}}  A \texttt{cons} cell can be thought of as an
object consisting of two \emph{fields} called the \texttt{car} field
and the \texttt{cdr} field respectively.%
\footnote{The names \texttt{car} and \texttt{cdr} come from the
  machine instructions on one of the first computers on which \lisp{}
  was implemented.  These instructions would return the \emph{contents
    of the address part of the register} and the \emph{contents of the
    decrement part of the register} respectively.}
\refFig{fig-cons-cell} illustrates a \texttt{cons} cell.

\begin{figure}
\begin{center}
\inputfig{fig-cons-cell.pdf_t}
\end{center}
\caption{\label{fig-cons-cell}
Representation of a \texttt{cons} cell.}
\end{figure}

More specifically, a \emph{list} is a sequence of \texttt{cons} cells,
linked by their \texttt{cdr} fields.  What is contained in the
\texttt{car} field is of no importance for the list itself.

\cl{} distinguishes between three basic types of lists:

\begin{itemize}
\item \emph{Proper} lists.  A list is a proper list when the last
  \texttt{cons} cell in the sequence contains the symbol \texttt{nil}
  in its \texttt{cdr} field. 
\item \emph{Dotted} lists.  A list is a dotted list when the last
  \texttt{cons} cell in the sequence contains an \emph{atom}
  in its \texttt{cdr} field.  An atom is any \cl{} object other than a
  \texttt{cons} cell.
\item \emph{Circular} lists.  A list is a circular list when the
  \texttt{cdr} field of some \texttt{cons} cell in the sequence
  contains a reference to a \texttt{cons} cell that precedes it in the
  sequence. 
\end{itemize}

In many situations where a proper list is required as argument to some
function, the \cl{} standard does not oblige the implementation to
check whether the function was given a dotted list or a circular
list.  As a result, some functions in some implementations may go into
an infinite computation when given a circular list.
