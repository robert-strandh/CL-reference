\chapter{Trees}
\label{chap-trees}

When \texttt{cons} cells are used to build more complex structures
than lists, the resulting structure is often called a \emph{tree}.
Then a list or an individual \texttt{cons} cell becomes a special case
of a tree.

In this chapter, we present functions that are meant to work on trees
other than lists and individual \texttt{cons} cells.

Several functions in this chapter are described in terms of
combinations of the simpler functions \texttt{car} and \texttt{cdr}.
These two functions are described in detail in
\refChap{chap-basic-cons-functions}.  In particular, recall that these
two functions return \texttt{nil} when given \texttt{nil} as an
argument, and that if they are given neither \texttt{nil} nor a
\texttt{cons} cell, then an error may or may not be signaled,
depending on whether the code is \emph{safe}.  See
\refChap{chap-basic-cons-functions} for details.

\Defun {caar} {tree}

The call (\texttt{caar} \textit{tree}) is equivalent to the call
(\texttt{car} (\texttt{car} \textit{tree})).

\Defun {cdar} {tree}

The call (\texttt{cdar} \textit{tree}) is equivalent to the call
(\texttt{cdr} (\texttt{car} \textit{tree})).

\Defun {caaar} {tree}

The call (\texttt{caaar} \textit{tree}) is equivalent to the call
(\texttt{car} (\texttt{car} (\texttt{car} \textit{tree}))).

\Defun {caadr} {tree}

The call (\texttt{caadr} \textit{tree}) is equivalent to the call
(\texttt{car} (\texttt{car} (\texttt{cdr} \textit{tree}))).

\Defun {cadar} {tree}

The call (\texttt{cadar} \textit{tree}) is equivalent to the call
(\texttt{car} (\texttt{cdr} (\texttt{car} \textit{tree}))).
