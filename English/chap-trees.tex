\chapter{Trees}
\label{chap-trees}

When \texttt{cons} cells are used to build more complex structures
than lists, the resulting structure is often called a \emph{tree}.
Then a list or an individual \texttt{cons} cell becomes a special case
of a tree.

In this chapter, we present functions that are meant to work on trees
other than lists and individual \texttt{cons} cells.

Several functions in this chapter are described in terms of
combinations of the simpler functions \texttt{car} and \texttt{cdr}.
These two functions are described in detail in
\refChap{chap-basic-cons-functions}.  In particular, recall that these
two functions return \texttt{nil} when given \texttt{nil} as an
argument, and that if they are given neither \texttt{nil} nor a
\texttt{cons} cell, then an error may or may not be signaled,
depending on whether the code is \emph{safe}.  See
\refChap{chap-basic-cons-functions} for details.

\Defun {caar} {tree}

The call (\texttt{caar} \textit{tree}) is equivalent to the call
(\texttt{car} (\texttt{car} \textit{tree})).

\Defun {cdar} {tree}

The call (\texttt{cdar} \textit{tree}) is equivalent to the call
(\texttt{cdr} (\texttt{car} \textit{tree})).

\Defun {caaar} {tree}

The call (\texttt{caaar} \textit{tree}) is equivalent to the call
(\texttt{car} (\texttt{car} (\texttt{car} \textit{tree}))).

\Defun {caadr} {tree}

The call (\texttt{caadr} \textit{tree}) is equivalent to the call
(\texttt{car} (\texttt{car} (\texttt{cdr} \textit{tree}))).

\Defun {cadar} {tree}

The call (\texttt{cadar} \textit{tree}) is equivalent to the call
(\texttt{car} (\texttt{cdr} (\texttt{car} \textit{tree}))).

\Defun {cdaar} {tree}

The call (\texttt{cdaar} \textit{tree}) is equivalent to the call
(\texttt{cdr} (\texttt{car} (\texttt{car} \textit{tree}))).

\Defun {cdadr} {tree}

The call (\texttt{cdadr} \textit{tree}) is equivalent to the call
(\texttt{cdr} (\texttt{car} (\texttt{cdr} \textit{tree}))).


\Defun {cddar} {tree}

The call (\texttt{cddar} \textit{tree}) is equivalent to the call
(\texttt{cdr} (\texttt{cdr} (\texttt{car} \textit{tree}))).

\Defun {caaaar} {tree}

The call (\texttt{caaaar} \textit{tree}) is equivalent to the call
(\texttt{car} (\texttt{car} (\texttt{car} (\texttt{car}
\textit{tree})))).

\Defun {caaadr} {tree}

The call (\texttt{caaadr} \textit{tree}) is equivalent to the call
(\texttt{car} (\texttt{car} (\texttt{car} (\texttt{cdr}
\textit{tree})))).

\Defun {caadar} {tree}

The call (\texttt{caadar} \textit{tree}) is equivalent to the call
(\texttt{car} (\texttt{car} (\texttt{cdr} (\texttt{car}
\textit{tree})))).

\Defun {caaddr} {tree}

The call (\texttt{caaddr} \textit{tree}) is equivalent to the call
(\texttt{car} (\texttt{car} (\texttt{cdr} (\texttt{cdr}
\textit{tree})))).

\Defun {cadaar} {tree}

The call (\texttt{cadaar} \textit{tree}) is equivalent to the call
(\texttt{car} (\texttt{cdr} (\texttt{car} (\texttt{car}
\textit{tree})))).

\Defun {cadadr} {tree}

The call (\texttt{cadadr} \textit{tree}) is equivalent to the call
(\texttt{car} (\texttt{cdr} (\texttt{car} (\texttt{cdr}
\textit{tree})))).

\Defun {caddar} {tree}

The call (\texttt{caddar} \textit{tree}) is equivalent to the call
(\texttt{car} (\texttt{cdr} (\texttt{cdr} (\texttt{car}
\textit{tree})))).

\Defun {cdaaar} {tree}

The call (\texttt{cdaaar} \textit{tree}) is equivalent to the call
(\texttt{cdr} (\texttt{car} (\texttt{car} (\texttt{car}
\textit{tree})))).

\Defun {cdaadr} {tree}

The call (\texttt{cdaadr} \textit{tree}) is equivalent to the call
(\texttt{cdr} (\texttt{car} (\texttt{car} (\texttt{cdr}
\textit{tree})))).

\Defun {cdadar} {tree}

The call (\texttt{cdadar} \textit{tree}) is equivalent to the call
(\texttt{cdr} (\texttt{car} (\texttt{cdr} (\texttt{car}
\textit{tree})))).

\Defun {cdaddr} {tree}

The call (\texttt{cdaddr} \textit{tree}) is equivalent to the call
(\texttt{cdr} (\texttt{car} (\texttt{cdr} (\texttt{cdr}
\textit{tree})))).

\Defun {cddaar} {tree}

The call (\texttt{cddaar} \textit{tree}) is equivalent to the call
(\texttt{cdr} (\texttt{cdr} (\texttt{car} (\texttt{car}
\textit{tree})))).

\Defun {cddadr} {tree}

The call (\texttt{cddadr} \textit{tree}) is equivalent to the call
(\texttt{cdr} (\texttt{cdr} (\texttt{car} (\texttt{cdr}
\textit{tree})))).

\Defun {cdddar} {tree}

The call (\texttt{cdddar} \textit{tree}) is equivalent to the call
(\texttt{cdr} (\texttt{cdr} (\texttt{cdr} (\texttt{car}
\textit{tree})))).

\Defun {copy-tree} {tree}

Return a copy of the tree \textit{tree}.

The semantics of \texttt{copy-tree} are as if it were defined like
this:

\begin{verbatim}
(defun copy-tree (tree)
  (if (atom tree)
      tree
      (cons (copy-tree (car tree)) (copy-tree (cdr tree)))))
\end{verbatim}

In particular, no attempt is made to detect circularity or to preserve
sharing of substructure.

\Defun {tree-equal} {tree-1 tree-2 \key test test-not}

Return a generalized Boolean value indicating whether \textit{tree-1}
is equal to \textit{tree-2}.

The keyword arguments \textit{test} and \textit{test-not} determine
whether two \emph{atoms} in the tree are equal.  By default, the value
of \textit{test} is \texttt{eql} and \textit{test-not} is absent.

If a value for either \textit{test} or for \textit{test-not} is given,
then it must be a designator for a function of two arguments that
returns a generalized Boolean.  The first argument supplied to the
function will be an atom from \textit{tree-1} and the second argument
supplied will be an atom from \textit{tree-2}.  If \textit{test} was
supplied, the two atoms are considered to be equal if and only if the
test returns a \emph{true} value.  If \textit{test-not} was supplied,
the two atoms are considered to be equal if and only if the test
returns \emph{false} (i.e., \texttt{nil}).

The consequences are unspecified if values are supplied both for
\textit{test} and for \textit{test-not}.  Most implementations will
probably signal an error.

Recall that a designator for a function is either a function or a
symbol.  If it is a symbol, the function denoted is the function named
by that symbol in the global environment.  The consequences are
undefined if a symbol is used as a function designator, but that
symbol does not name a function in the global environment.

The consequences are undefined if both \textit{tree-1} and
\textit{tree-2} have circular structures in them.  If only one of them
has circular structure in it, then the result is well defined, because
then ultimately an atom of the non-circular tree will be tested
against some object in the circular tree.  If that atom is tested
against an atom in the circular tree, the result is determined by the
\textit{test} or the \textit{test-not} argument.  If, on the other
hand, that atom is tested against a \texttt{cons} cell in the circular
tree, the result is always \emph{false}.
